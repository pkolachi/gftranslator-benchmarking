\documentclass[a4paper, 11pt]{article}
\topmargin-2.0cm

\usepackage{fancyhdr}
\usepackage{lastpage}
\usepackage{extramarks}
\usepackage{pagecounting}
\usepackage{color}
\usepackage[round]{natbib}
\usepackage{lipsum}
\usepackage{enumerate}
%\usepackage{tipa}
%\usepackage{gb4e}
\usepackage{graphicx}
\usepackage{amsmath, amsthm}
%\usepackage{colortbl}
\usepackage{caption}
\usepackage[a4paper]{geometry}
\usepackage{url}

\advance\oddsidemargin-0.35in
\advance\evensidemargin-0.65in
\textheight9.5in
\textwidth6.5in

\newcommand{\hsp}{\hspace*{\parindent}}
\definecolor{gray}{rgb}{0.4,0.4,0.4}

\newcommand\blfootnote[1]{
  \begingroup
  \renewcommand\thefootnote{}\footnote{#1}
  \addtocounter{footnote}{-1}
  \endgroup
}

%\sectionfont{\large}

\begin{document}

\pagestyle{fancy}
\lhead{\textcolor{gray}{Prasanth Kolachina}}
\chead{\textcolor{gray}{}}
\rhead{\textcolor{gray}{REMU}}
\lfoot{\textcolor{gray}{prasanth.kolachina@cse.gu.se}}
\rfoot{\thepage}
\renewcommand{\headrulewidth}{0.5pt} 
\renewcommand{\footrulewidth}{0.5pt} 
\fancyfoot[C]{\footnotesize \textcolor{gray}{}} 

\centerline{ {\Large MT Error Analysis: A case-study towards Hybrid MT} }
\vspace*{1cm}

There have been a number of hand-crafted \textit{computational grammar} development projects in the last
few decades attempting to build comprehensive wide-coverage grammars for natural languages using different
grammatical formalisms. Such formalisms have been designed to be richer than context-free grammars
in terms of their generative capacity. While efforts in defining these formalisms have contributed
to grammars with detailed linguistic analysis, such grammars also lack the distributional information
necessary for disambiguation tasks such as parsing. Alternatively, grammars constructed with necessary 
distributional information from annotated corpora like treebanks have shown to be effective in a wide 
variety of NLP applications, but are typically not linguistically interesting. 

However, these efforts to construct grammars from annotated corpora are often interleaved with 
language-specific and annotation-specific information to extract linguistic units of grammars. Such
annotation-specific information can be abstracted away during grammar extraction, allowing uniform
extraction of grammars for multiple languages. This has been verified in the case of context-free grammars
where language-independent methods to construct grammars from corpora have been proposed over time.
In my talk, I will address this issue in the context of Tree Adjoining Grammars. TAG grammars, proposed by
Joshi et. al (1976) have been developed for a wide range of languages and put to use in a multitude of
NLP applications ranging from parsing to generation.

I propose a `normative' grammar extraction procedure to extract multi-lingual TAG grammars by seperating out
language- and annotation-specific details out of the extraction procedure. As part of this, I will address
the specific problem of inducing argument/adjunct distinction in syntactic structures without
using annotation-specific details. I will present the results of my experiments on the Swedish treebank
\textit{Talbanken}, and show that the procedure can indeed work in an annotation-neutral manner. The results
show that the extracted grammars can serve as a first-order approximation to hand-crafted grammars useful
in creating wide-coverage grammars. 

%If you want references on a separate page, uncomment the following command.
%\clearpage
%\begin{small}
%\bibliographystyle{plainnat}
%\bibliography{references}
%\end{small}

\end{document}
